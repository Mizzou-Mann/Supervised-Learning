% assignment_2.tex
% CS 8725 - Supervised Learning (Fall 2015)
%     University of Missouri-Columbia
%             Chanmann Lim
%            September 2015

\documentclass[a4paper]{article}

\usepackage[margin=1 in]{geometry}
\usepackage{amsmath}
\usepackage{float}

\everymath{\displaystyle}
\DeclareMathOperator*{\argmax}{\arg\!\max}

\begin{document}
\title{CS 8725: Report for assignment 2}
\author{Chanmann Lim}
\date{September 9, 2015}
\maketitle

\paragraph{1.} Parameters list: \\

	$ P(Y) $
	
	$ P(X_1|Y = y), \qquad y \in \{T, F\} $
	
	$ P(X_i|Y = y) \sim N(\mu_{y_i}, \sigma_{y_i}^2), \qquad y \in \{T, F\}, \; 2 \leq i \leq d $
	
	The total number of parameters = $1 + 2 + 2 \times 2 \times (d-1) = 4d - 1 $.
	\begin{align}
		P(Y|X) &= \frac{P(X|Y) \cdot P(Y)}{P(X)} \\
			&= \frac{P(X_1|Y) \cdot \prod_{i=2}^d N(\mu_{i}, \sigma_{i}^2) \cdot P(Y)}{P(X)}
	\end{align} 

\paragraph{2. (a)} ~\\
	\begin{align}
		f_{NB}(Sunny, Windy) = \argmax_{Y} P(Sunny|Y) \cdot P(Windy|Y) \cdot P(Y)
	\end{align}
	Where $Y \in \{Hike, \neg Hike\}$ and 
	$$ P(Hike) = P(\neg Hike) = 0.5 $$
	\begin{align}
		f_{NB}(Sunny, Windy) = \argmax_{Y} P(Sunny|Y) \cdot P(Windy|Y)
	\end{align}
	
\paragraph{(b)} ~\\
	\begin{align}
		P(Sunny, Windy, Hike) &= P(Sunny, Windy | Hike) \cdot P(Hike)\\
			&= P(Sunny | Hike) \cdot P(Windy | Hike) \cdot P(Hike)\\
			&= 0.8 \times 0.4 \times 0.5\\
			&= 0.16
	\end{align}
	Similarly,
	\begin{align}
		P(Sunny, Windy, \neg Hike) &= P(Sunny | \neg Hike) \cdot P(Windy | \neg Hike) \cdot P(\neg Hike)\\
			&= 0.7 \times 0.5 \times 0.5\\
			&= 0.175
	\end{align}
	And the probability of error:
	\begin{align}
		P_e &= 1 - P(Correct) \\
			&= 1 - P(Y|Sunny, Windy) \\
			&= 1 - \frac{P(Sunny, Windy, Y) \cdot P(Y)}{P(Sunny, Windy)}
	\end{align}
	For the case when the weather is sunny and windy the error probability:
	\begin{align}
		P_e(Hike|Sunny, Windy) &= 1 - \frac{0.16}{0.16 + 0.175} \\
			&= 1 - 0.48 \\
			&= 0.52
	\end{align}
\end{document}