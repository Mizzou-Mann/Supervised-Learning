% assignment_2.tex
% CS 8725 - Supervised Learning (Fall 2015)
%     University of Missouri-Columbia
%             Chanmann Lim
%            September 2015

\documentclass[a4paper]{article}

\usepackage[margin=1 in]{geometry}
\usepackage{amsmath}
\usepackage{float}

\everymath{\displaystyle}
\DeclareMathOperator*{\argmax}{\arg\!\max}

\begin{document}
\title{CS 8725: Report for assignment 2}
\author{Chanmann Lim}
\date{September 9, 2015}
\maketitle

\paragraph{1.} For $Y \in \{T, F\}$, one parameter is needed to describe $P(Y)$, two parameters are needed to describe $P(X_1|Y)$ and four parameters $(\mu_{iY}, \sigma_{iY})$ are needed to describe $P(X_i|Y)$ for $2 \leq i \leq d$.\\

	$ P(Y) $
	
	$ P(X_1|Y), \qquad Y \in \{T, F\}, \; X_1 \in \{T, F\} $
	
	$ P(X_i|Y) \sim N(\mu_{iY}, \sigma_{iY}^2), \qquad Y \in \{T, F\}, \; 2 \leq i \leq d $
	
	The total number of parameters = $1 + 2 + 4 \times (d-1) = 4d - 1 $.	
	\begin{align}
		P(Y|X) &= \frac{P(X|Y) \cdot P(Y)}{P(X)} \\
			&\propto P(X|Y) \cdot P(Y) \\
			&= P(X_1|Y) \cdot \prod_{i=2}^d N(x_i; \mu_{iY}, \sigma_{iY}^2) \cdot P(Y)
	\end{align}
	Where, $N(x_i; \mu_{iY}, \sigma_{iY}^2) = \frac{1}{\sqrt{2\pi}\sigma_{iY}}e^{-\frac{(x_i-\mu_{iY})^2}{2\sigma_{iY}}}$

\paragraph{2. (a)} Naive Bayes decision rule is the function that maximizes the posterior probability however the denominator of the posterior probability is merely the normalization term to satisfy the property of probability and it does not depend on $y$ thus it can be dropped.\\
	\begin{align}
		f_{NB}(Sunny, Windy) = \argmax_{y} P(Sunny|y) \cdot P(Windy|y) \cdot P(y)
	\end{align}
	Where $Y \in \{Hike, \neg Hike\}$ and 
	$$ P(Y = Hike) = P(Y = \neg Hike) = 0.5 $$
	$P(y)$ can also be dropped.
	\begin{align}
		f_{NB}(Sunny, Windy) = \argmax_{y} P(Sunny|y) \cdot P(Windy|y)
	\end{align}
	Therefore, predict Alice and Bob go hiking if $P(Sunny|Hike) \times P(Windy|Hike) > P(Sunny|\neg Hike) \times P(Windy|\neg Hike)$ and else otherwise.
	
\paragraph{(b)} ~\\
	\begin{align}
		P(Sunny, Windy, Hike) &= P(Sunny, Windy | Hike) \cdot P(Hike)\\
			&= P(Sunny | Hike) \cdot P(Windy | Hike) \cdot P(Hike)\\
			&= 0.8 \times 0.4 \times 0.5\\
			&= 0.16
	\end{align}
	Similarly,
	\begin{align}
		P(Sunny, Windy, \neg Hike) &= P(Sunny | \neg Hike) \cdot P(Windy | \neg Hike) \cdot P(\neg Hike)\\
			&= 0.7 \times 0.5 \times 0.5\\
			&= 0.175
	\end{align}
	And the probability of error:
	\begin{align}
		P_e &= 1 - P(Correct) \\
			&= 1 - P(Y|Sunny, Windy) \\
			&= 1 - \frac{P(Sunny, Windy, Y) \cdot P(Y)}{P(Sunny, Windy)}
	\end{align}
	For the case when the weather is sunny and windy the error probability:
	\begin{align}
		P_e(Hike|Sunny, Windy) &= 1 - \frac{0.16}{0.16 + 0.175} \\
			&= 1 - 0.48 \\
			&= 0.52
	\end{align}
\end{document}